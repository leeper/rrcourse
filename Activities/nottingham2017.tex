\documentclass[12pt, a4paper]{article}
\usepackage[margin=2cm]{geometry}
\usepackage{setspace}
\usepackage{alltt}
\usepackage{verbatim}
\usepackage{booktabs}
\usepackage{mdwlist}
\usepackage[hidelinks]{hyperref}

\title{Reproducibility for Your Own Benefit}
\date{}

\begin{document}

\vspace{-4em}

\maketitle

\vspace{-3em}

Reproducible research should be for the researcher's benefit, not a distraction. With that idea in mind, this document walks through the use GitHub, GitKraken, and Rmarkdown even if your only collaboration is with their ``future self''.

\section{GitHub}

GitHub is a widely used platform for open sharing of data, software, analysis code, and other materials. It uses software called \textbf{git}, which a powerful tool for organizing, collaborating on, and version controlling a project. We'll use GitHub to host our example project from OSF and then use git on our own machines to showcase some of its features.

\begin{enumerate*}
\item Create a GitHub account at \url{https://github.com}
\item Once setup, push the green \texttt{New repository} button
\item Given the repository a name. It can be anything. ``nottingham'', ``first-project'', etc.
\item You can give it a description if you want.
\item Click \texttt{Create repository}. You're all setup!
\end{enumerate*}

\section{GitKraken}

\textbf{git} is the software that powers GitHub and which we will use to \textit{version control} our project content, experiment with that content, and possibly introduce collaboration. There are lots of different tools for working with git, the simplest of which is a command-line interface that can be used via Terminal, Command Prompt, bash, etc.\footnote{You can find installation instructions for it here: \url{https://git-scm.com/downloads}} For today's workshop, however, we will use a (free) product called GitKraken that provides a point-and-click interface for git. Set it up as follows:

\begin{enumerate*}
\item Download GitKraken from \url{https://www.gitkraken.com/} and complete the installation
\item Open GitKraken and log in using your GitHub login details
\item Click \texttt{File > Init Repo}
\item Enter the folder containing the example project as the \texttt{New repository path}
\item Click \texttt{Create Repository}. You should see all of the project files in green near the centre of the screen.
\item On the left side, click the \texttt{+} symbol next to \texttt{Remote}
\item On the pop-up window, click \texttt{URL}.
\item For \texttt{Name}, type ``GitHub''. For \texttt{Pull URL}, type ``https://github.com/\texttt{user}/\texttt{project}''
\end{enumerate*}

Now we have a Github ``remote'' repository setup and the local git repository setup with GitKraken. This means we're ready to start working with our project.

\clearpage

\section{git Basics in GitKraken}

Let's put git to work for us identifying when files have changed, recording those changes as \textit{snapshots}, and navigating between versions. To that, we need to know some actions that git can perform. The key idea is that we will create snapshots of our project by \textbf{committing} files to the git repository and those changes will be reflected on Github (and to any collaborators) when we \textbf{push} them. Here's a glossary, which we'll walk through together:

\begin{itemize}
\item \textbf{stage}: select files to be recorded in a ``snapshot'' of the project
	\begin{itemize}
	\item \textbf{unstage}: remove files from the snapshot (but not from your computer)
	\end{itemize}
\item \textbf{commit}: record a permanent snapshot of the staged files, labelled with a ``commit message''
	\begin{itemize}
	\item \textbf{amend}: modify (typically the most recent) commit with new changes or commit message
	\end{itemize}
\item \textbf{branch}: produce a complete local copy of the project where changes can be made independently of the ``master'' branch
\item \textbf{merge}: update a branch with changes from another local branch (or a collaborator's changes from GitHub); you can change multiple branches independently.
\item \textbf{push}: send the project (any new commits) to a remote server (like GitHub)
\item \textbf{pull}: grab new commits from a remote server
\end{itemize}

GitKraken provides a point-and-click interface for all of these steps, but you can also execute them via the CLI commands \texttt{git add}, \texttt{git commit}, \texttt{git branch}, etc.

\subsection*{Test Yourself}

You get a hang of what git can offer by trying out the following.

\begin{enumerate*}
\item Stage and commit your files
\item Change one or more files in any way. See how those changes become ``unstaged''.
\item Create a new branch to contain these changes.
\item Stage and commit the changes in the new branch.
\item Push the new branch to GitHub. See that they show up there.
\item Merge the changes to the master branch in GitKraken.
\item Push the changes to the master branch on GitHub.
\item Change some files through the GitHub web interface. Pull them locally.
\end{enumerate*}

\subsection*{Key Steps}

Overall, the most essential steps of git are \texttt{pull > stage > commit > push}.

\noindent As a mnemonic device, maybe try ``\textbf{P}eter \textbf{S}hould \textbf{C}ome \textbf{P}romptly''.


\clearpage
\section{Reproducible documents}

Once you've mastered git for managing and tracking your files, another useful set of tools to know relate to reproducible, dynamic documents. We'll use Rmarkdown as an example for this.

\footnotesize 
\begin{verbatim}
---
title: "Money, Happiness and the Midlife Crisis"
author: "Richard Ball"
date: 2017-03-08
output: pdf_document
---

```{r data, echo=FALSE, results="hide"}
library("knitr")
opts_chunk$set(echo=FALSE)
library("rio")
library("ggplot2")
suppressPackageStartupMessages(library("stargazer"))
country <- rio::import("../Analysis Data/country-analysis.dta")
individual <- rio::import("../Analysis Data/individual-analysis.dta")
```

There are some descriptive statistics about the data:

```{r table1}
kable(aggregate(cbind(cm_satis, inc, exp) ~ CountryName, data = country, FUN = mean))
```

And the main regression are shown in the table.

```{r table2, results = "asis"}
stargazer(
  x1 <- lm(satis ~ age + age2, data = individual),
  x2 <- lm(satis ~ age + age2 + factor(country), data = individual),
  header = FALSE
)
```

```{r inlinenumbers, results="hide"}
ageatminsatis1 <- -coef(x1)["age"]/(2*coef(x1)["age2"])
ageatminsatis2 <- -coef(x2)["age"]/(2*coef(x2)["age2"])
```

Figures 1 and 2 show some patterns.

```{r figure1}
ggplot(country, aes(x = inc, y = cm_satis)) + 
  geom_point() + 
  geom_text(aes(label = CountryName))
```

```{r figure1}
ggplot(country, aes(x = exp, y = cm_satis)) + 
  geom_point() + 
  geom_text(aes(label = CountryName))
```
  
To summarize, here are some main results. To summarize, here are some main results. 
The estimated minimum SWB without country fixed effects is `r ageatminsatis1`.
By contrast, the estimated minimum SWB with country fixed effects is `r ageatminsatis2`.
\end{verbatim}


\end{document}