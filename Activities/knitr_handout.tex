\documentclass[11pt, a4paper]{article}
\usepackage[margin=2cm]{geometry}
\usepackage{setspace}
\usepackage{alltt}
\usepackage{verbatim}
\usepackage{booktabs}
\usepackage{mdwlist}
\usepackage[hidelinks]{hyperref}

\title{Notes on Reproducible Research with knitr}
\date{}

\begin{document}

\maketitle
\onehalfspacing

\section*{General knitr Chunk Options}

How each code chunk is handled during \texttt{knit}-ing depends on \textit{chunk options} that control the behavior and appearance of each chunk. The main chunk options are:

\begin{itemize*}
    \item \texttt{eval}: Whether the chunk should be evaluated
    \item \texttt{echo}: Whether the code contained in the chunk should be displayed in the PDF
    \item \texttt{results}: 
        \begin{itemize*}
            \item ``markup'' (default): Results are displayed like in the R console
            \item ``asis'': Used for any chunk containing LaTeX output (e.g., tables)
            \item \texttt{hold}: All results are held until the end of the chunk (versus displayed line-by-line)
            \item ``hide'': Results are not displayed
        \end{itemize*}
    \item \texttt{warning} and \texttt{message}: Whether to suppress warnings and messages
    \item \texttt{tidy} and \texttt{highlight}: Whether R code should be tidied and syntax highlighted, respectively
\end{itemize*}

\noindent Chunk options can be set in each chunk:

\begin{alltt}
<<a, eval = TRUE, echo = FALSE>>=
2 + 2
@
\end{alltt}

\vspace{1em}
\noindent Or, they can be set globally inside a ``setup'' chunk that applies options to all other chunks:
\begin{alltt}
<<a, eval = TRUE, echo = FALSE>>=
opts\_chunk\$set(echo = FALSE, cache = TRUE, message = FALSE)
@
\end{alltt}


\vspace{1em}
\section*{General knitr Package Options}

knitr also allows you to specify some \textit{package options} to control the behavior of knitr in general. These are probably not helpful to beginners, but may be useful for advanced users. You can find a list of these options at \href{http://yihui.name/knitr/options#package_options}{http://yihui.name/knitr/options\#package\_options}.


\clearpage
\section*{knitr Plotting}

There are two ways to include plots in a knitr workflow. One involves using manual includes of plots that are generated in a chunk, saved locally, and then included using \verb|\includegraphics{}|, for example:

\begin{verbatim}
<<>>=
pdf('figures/barplot.pdf')
barplot(mtcars$cyl)
dev.off()
@

\begin{figure}
\caption{A barplot}
\label{fig:barplot}
\includegraphics{figures/barplot}
\end{figure}
\end{verbatim}

\noindent The alternative is to specify figure-related options and let knitr handle the details:

\begin{verbatim}
<<fig.cap='A barplot', fig.lp='fig:barplot'>>=
barplot(mtcars$cyl)
@
\end{verbatim}

\noindent The figure-relevant options useful for controlling the behavior of a knitr plot chunk are:

\begin{itemize}
    \item \texttt{fig.path}: The path to store and retrieve figures from (default is \texttt{./figure})
    \item \texttt{fig.show}
        \begin{itemize*}
            \item \texttt{asis}: All plots are rendered line-by-line
            \item \texttt{hold}: All plots are held until the end of the chunk
            \item \texttt{animate}: Plot(s) converted into an animation (doesn't work in all PDF viewers)
            \item \texttt{hide}: Plot(s) not shown
        \end{itemize*}
    \item \texttt{fig.height} and \texttt{fig.width}
    \item \LaTeX-related options
        \begin{itemize*}
            \item \texttt{fig.env}: \LaTeX figure environment to use
            \item \texttt{fig.cap}: \LaTeX caption
            \item \texttt{fig.lp}: \LaTeX label (for cross-referencing)
        \end{itemize*}
    \item \texttt{dev}
        \begin{itemize*}
            \item \texttt{dev.args}: A list containing arguments passed to the graphics device function
        \end{itemize*}
\end{itemize}


\clearpage
\section*{\LaTeX{} Tables}

Tables are created in \LaTeX{} using the \texttt{tabular} environment:\\

\vspace{2em}

\begin{minipage}{.45\textwidth}
\begin{verbatim}
\begin{tabular}{l r r} \toprule
Header 1 & Header 2 & Header 3\\
\midrule
Row 1 \\
Row 2 \\ \bottomrule
\end{tabular}
\end{verbatim}

\end{minipage}
\begin{minipage}{.45\textwidth}

\begin{tabular}{l r r} \toprule
Header 1 & Header 2 & Header 3\\ \midrule
Row 1 & & \\
Row 2 & & \\ \bottomrule
\end{tabular}

\end{minipage}

\vspace{2em}

\noindent By default, a \texttt{tabular} environment will appear in the document exactly where in occurs in the \LaTeX{} source. In most academic writing, we'll additionally want to include this \texttt{tabular} environment inside a \texttt{table} environment, which is a \textit{float} environment that will cause the table to be displayed at the top or bottom of an appropriate page. This matters because we can only cross-reference tables if they are in a float environment and contain a \texttt{label}. For example:

\begin{verbatim}
\begin{table}
\caption{This is the table's title}
\label{tab:simple}
\begin{tabular}{l r r} \toprule
Header 1 & Header 2 & Header 3\\
\midrule
Row 1 \\
Row 2 \\ \bottomrule
\end{tabular}
\end{table}
\end{verbatim}

\noindent Using that, we can then refer to the table anywhere in our document using \verb|Table \ref{tab:simple}| and it will be replaced by ``Table 1'' (and the number will be automatically updated depending on the table's location in the document).

\subsection*{Creating Tables}

You can mock-up your own \LaTeX{} tables by-hand, but it is much easier to use R packages to produce that \LaTeX{} code automatically. This applies even if you don't plan to use knitr in your workflow. You can use R package to write \LaTeX{} output to a local file (e.g., \texttt{table.tex}) and then include that table in your article using: \verb|\input{table.tex}|, wherever you want the table to appear in your document.

\vspace{1em}
\noindent The following sections detail how to create tables using three popular packages. The \href{http://cran.r-project.org/web/views/ReproducibleResearch.html}{Reproducible Research TaskView on CRAN} lists functionality found in other packages for creating \LaTeX{}-formatted output.

\clearpage
\subsection*{\texttt{kable}: Simple tables}

\texttt{kable} from the knitr package is an easy way to create simple tables without a float environment. For example, we can use \texttt{kable} to print a simple correlation matrix with something like:\\
\verb|kable(cor(mtcars), "latex")|

\noindent \texttt{kable} includes a few options to control appearance:
\begin{itemize}
\item \texttt{format}: This will always be ``latex'' for our purposes
\item \texttt{digits}: How many digits to print in a numeric table
\item \texttt{row.names}, \texttt{col.names}: Whether to include row or column names for a table
\item \texttt{align}: A vector of alignments for the table columns
\end{itemize}

\vspace{1em}
\subsection*{\texttt{xtable}: Generic table creation}

\texttt{xtable} from the \texttt{xtable} package is a much more powerful (and thus complicated) function for create simple tables (with or without float environments). For example we can use it to create one- or more-way tabulations using, e.g., \verb|xtable(table(mtcars$cyl))| or \verb|xtable(with(mtcars, table(cyl, gear)))|. There are a number of options that can be specified for \texttt{xtable}:

\begin{itemize}
\item \texttt{caption}: A \LaTeX{} caption
\item \texttt{label}: A \LaTeX{} label for cross-referencing
\item \texttt{align}: A vector of column alignments
\item \texttt{digits}: How many digits to print in a numeric table
\end{itemize}

\noindent \texttt{xtable} is also a little confusing because it includes options for its \texttt{print} method that further control the appearance of tables, such as the float environment for the table. These options are called within \texttt{print}, but outside of \texttt{xtable}, e.g.: \verb|print(xtable(table(mtcars$cyl)), floating = FALSE)| produces a table with no float environment.

\begin{itemize}
\item \texttt{file}: A file path to write the table to
\item \texttt{floating}: Whether to include the table in a float environment
\item \texttt{floating.environment}: What float environment to use (\texttt{table}, by default)
\item \texttt{hline.after}: Where to place horizontal lines in the table
\item \texttt{NA.string}: How missing values (\texttt{NA}'s) should be displayed
\item \texttt{include.rownames}, \texttt{include.colnames}: Whether to include row and column names
\item \texttt{sanitize.text.function}: A function to handle \LaTeX{}-incompatible text strings
\item \texttt{booktabs}: Whether to use the \LaTeX{} \texttt{booktabs} package for more attractive tables
\end{itemize}


\pagebreak
\subsection*{\texttt{stargazer}: Tables for model objects}

\texttt{stargazer} from the stargazer package is useful for creating tables for model objects (e.g., regression results). While \texttt{xtable} can also do this, \texttt{stargazer} does it more elegantly by aligning variables from multiple models and better controlling printing of supplemental information (e.g., goodness-of-fit statistics). This produces a simple two-model output:\\ \verb|stargazer(lm(mpg ~ cyl, data = mtcars), lm(mpg ~ cyl + hp, data = mtcars))|\\Output can be controlled using many options, some of which are shown below:

\begin{itemize}
\item \texttt{title}: A \LaTeX{} caption
\item \texttt{label}: A \LaTeX{} label for cross-referencing
\item \texttt{out}: A file path to print the table to
\item \texttt{column.labels}: Labels for the model columns
\item \texttt{covariate.labels}: Labels to replace covariate variable names. (Note: This requires some care.)
\item \texttt{digits}: How many digits to print in the table
\item \texttt{float}: Whether to include the table in a float environment
\item \texttt{float.env}: What float environment to use (\texttt{table}, by default)
\item \texttt{model.names}: Whether to print type of model for each column
\item \texttt{model.numbers}: Whether to number the models
\item \texttt{dep.var.caption}: The caption for the dependent variables
\item \texttt{dep.var.labels.include}: Whether to include dependent variable labels
\item \texttt{multi.column}: Whether to span column headers across like columns
\item \texttt{keep.stat}: A vector of named statistics to include in the table:\\``all'', all statistics; ``adj.rsq'', adjusted R-squared; ``aic'', Akaike Information Criterion; ``bic'', Bayesian Information Criterion; ``chi2'', chi-squared; ``f'', F statistic; ``ll'', log-likelihood; ``logrank'', score (logrank) test; ``lr'', likelihood ratio (LR) test; ``max.rsq'', maximum R-squared; ``n'', number of observations; ``null.dev'', null deviance; ``Mills'', Inverse Mills Ratio; ``res.dev'', residual deviance; ``rho'', rho; ``rsq'', R-squared; ``scale'', scale; ``theta'', theta; ``ser'', standard error of the regression (i.e., residual standard error); ``sigma2'', sigma squared; ``ubre'', Un-Biased Risk Estimator; ``wald'', Wald test.
\vspace{1em}
\item \texttt{style}: A character string naming a journal style. Includes some common social science journal formats.
\end{itemize}


\clearpage
\section*{Working with knitr}

The best way to learn knitr is to use it. I would recommend you try one of the following activities:

\begin{enumerate}
    \item Migrate one of your existing projects to knitr, either:
        \begin{itemize}
            \item Move R code into knitr code chunks
            \item Move all code into one chunk and configure automated \LaTeX{} includes for the results.
        \end{itemize}
    \item Setup the knitr template for a new project you haven't started
    \item Try creating a knitr document for a toy analysis project
\end{enumerate}

\vspace{1em}
\subsection*{Ideas for Toy Analysis Project}

Use one of R's built-in datasets (just type \texttt{data()} in the RStudio console) to conduct a simple analysis in a knitr workflow. 

\begin{itemize}
\item Possible datasets might include: \texttt{iris}, \texttt{mtcars}, \texttt{ChickWeight}, and \texttt{infert}
\item Run descriptive analyses that can be summarized in tables and included using \texttt{kable} or \texttt{xtable}
\item Run basic plots (\texttt{hist}, \texttt{barplot}, \texttt{plot}) and include those using knitr plot options
\item Run some regression models (e.g., \texttt{lm}, \texttt{glm}, etc.) and include the results using \texttt{stargazer}
\item Reference tables and figures in the text using \LaTeX{} cross-references (i.e., \verb|\ref{}|)
\item Reference particular analytic results using in-line code (i.e., \verb|\Sexpr{}|)
\end{itemize}

\noindent Here are some basic code examples to get you started using these datasets:

\singlespacing
\begin{verbatim}
# basic summary statistics
head(iris) # some of the raw data
summary(iris) # variable descriptive statistics

# plots
plot(infert[,1:4]) # multivariate correlation plot
hist(infert$age) # histogram of age
plot(density(iris$Sepal.Length)) # density plot
with(iris, plot(Sepal.Length, Sepal.Width)) # scatterplot

# tables
cor(iris[,1:4])
table(infert$education)
with(infert, table(age, education))
with(infert, table(education, induced))

# models
aov(Sepal.Length ~ Species, data = iris) # ANOVA (use `xtable`)
lm(Sepal.Length ~ Sepal.Width * Species, data = iris) # OLS
glm(case ~ education + age + induced + spontaneous, data = infert) # Logit

\end{verbatim}


\end{document}